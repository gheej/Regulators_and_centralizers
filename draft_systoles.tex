\documentclass[a4paper]{article}

\usepackage[utf8]{inputenc}
\usepackage[T1]{fontenc}
\usepackage{textcomp}
\usepackage[english]{babel}
\usepackage{amsmath, amssymb, amsthm, mathtools}
\usepackage{hyperref} % clickable links in the document
\usepackage{url} % clickable links in the document
\usepackage{enumitem} % (a), (b), etc labels

\usepackage{array}

\usepackage{xcolor} % support of \red text
\newcommand{\red}{\textcolor{red}} % support of \red text

\usepackage{listings}

% Define SageMath style for highlighting
\lstdefinestyle{sagemath}{
    language=Python, % SageMath is based on Python
    basicstyle=\ttfamily\footnotesize,
    commentstyle=\color{gray},
    keywordstyle=\color{blue},
    stringstyle=\color{red},
    frame=single,
    breaklines=true,
    showstringspaces=false,
    tabsize=4,
    captionpos=b,
    morekeywords={R, PolynomialRing, QQ} % Add SageMath specific keywords here
}

\newtheorem{Thm}{Theorem}[section]
\newtheorem{Lem}[Thm]{Lemma}
\newtheorem{Def}[Thm]{Definition}
\newtheorem{Cor}[Thm]{Corollary}
\newtheorem{Prop}[Thm]{Proposition}
\newtheorem{Rem}[Thm]{Remark}
\newtheorem{Ex}[Thm]{Example}
\newtheorem{Cla}[Thm]{Claim}
\newtheorem{Exer}[Thm]{Exercise}
\newtheorem{Prob}[Thm]{Problem}

\newcommand{\embeds}{\hookrightarrow}
\newcommand{\projects}{\twoheadrightarrow}
\newcommand{\eqdef}{\stackrel{\mathrm{def}}{=}}

\DeclareMathOperator{\assign}{\coloneqq}        % Assingment operator
\newcommand{\R}{\mathbb{R}}        % Real
\newcommand{\Q}{\mathbb{Q}}        % rationals
\newcommand{\C}{\mathbb{C}}        % Complex
\renewcommand{\P}{\mathbb{P}}        % Projective
\newcommand{\A}{\mathbb{A}}        % Affine
\renewcommand{\O}{\mathcal{O}}        % Structure sheaf
\newcommand{\Id}{\mathbf{Id}}        % Identity
\newcommand{\SL}{\mathbf{SL}_3(\mathbb{Z})}        % SL_3(Z)
\newcommand{\GL}{\mathbf{GL}_3} % GL_3
\newcommand{\PGL}{\mathbf{GL}_3} % PGL_3
\newcommand{\Mat}{\mathbf{Mat}_3(\mathbb{C})}        % Mat_3( C )
\newcommand{\Matn}{\mathbf{Mat}_n(\mathbb{C})}        % Mat_n( C )
\newcommand{\MatZ}{\mathbf{Mat}_3(\mathbb{Z})}        % Mat_3( ZZ )
\newcommand{\MatnZ}{\mathbf{Mat}_n(\mathbb{Z})}        % Mat_n( ZZ )
\newcommand{\SLp}{\Gamma_p}        % a congruence subgroup
\newcommand{\phip}{\phi_p}        % the golden ratio mod(p)
\newcommand{\hence}{\Rightarrow}        % hence
\newcommand{\minpoly}{\mathrm{minpoly}}        % minimal polynomial

\newcommand{\SLn}{\mathbf{SL}_n(\mathbb{Z})}        % SL_n(Z)

\DeclareMathOperator{\Spec}{Spec}        % Spec
\DeclareMathOperator{\Proj}{Proj}        % Proj
\DeclareMathOperator{\Reg}{Reg}        % Regulator
\DeclareMathOperator{\disc}{disc}        % discriminant
\DeclareMathOperator{\Hom}{Hom}        % Hom
\DeclareMathOperator{\Aut}{Aut}        % Aut


\begin{document}
\title{A sample computation}	
\author{Altan Erdnigor}
\maketitle

\tableofcontents

\section{Notation}
\begin{itemize}
\item $p > 2$ a prime number.
\item $\SL$ the special linear group over $\mathbb{Z}$.
\item $\SLp$ the $p$th congruence subgroup of $\SL$.
\item $Z_G(x)$ the centralizer of $x \in G$.
\item 
\begin{equation}
\label{matrix_small}
A = 
\begin{pmatrix}
0 & 0 & 1 \\
1 & 0 & 0 \\
0 & 1 & -p^2
\end{pmatrix}
\in \SL
\end{equation}
\item 
\begin{equation}
\label{matrix_big}
\tilde A = \Id + p A =
\begin{pmatrix}
1 & 0 & p \\
p & 1 & 0 \\
0 & p & 1 - p^3
\end{pmatrix}
\in \SLp
\end{equation}

\item If $C$ is a matrix, $\chi_C(\lambda) \assign \det(\lambda \Id - C)$ is the characteristics polynomial.
\item $ f(t) \assign \chi_A(t) = t^3 + p^2 t^2 - 1$.

\end{itemize}

\section{Intro}
In this note we establish the following results:
\begin{enumerate}
\item
The regulator of $\tilde A$
grows as $\approx 3 \ln^2(p)$. 

That is, $\Reg ( \Q(\alpha)/\Q ) \approx 3 \ln^2(p) $ for $\alpha$ a root of $\chi_{\tilde A}$.

\item
The index of the centralizers
$ [Z_{\SL}(\tilde A) : Z_{\SLp}(\tilde A)] = 3$
does not depend on $p$.
\end{enumerate}
This might be interesting.

\section{Regulators}
We mimic the proofs from a Keigh Conrad's write-up on the Dirichlet unit theorem \cite{conraddirichlet}.


Let $\Q(\alpha)$ be the number field obtain by attaching the root of $f(t)$ to $\Q$.
\begin{Prop}
$\Q(\alpha) / \Q$ is a totally real number field of degree $3$.
\end{Prop}

\begin{proof}
First, $f(t)$ is irreducible as it has no rational roots: $f(1) = p^2 \ne 0, f(-1) = p^2 - 2 \ne 0 $.

The discriminant of $f(t)$ equals
$$\disc_f(p) = 4 p^6 - 27 .$$

The discriminant is always positive $\disc_f(p) > 0$, thus the number field is totally real.
\end{proof}


\begin{Prop}
\label{prop_multiplicative_group_structure}
$\mathbb{Z}[\alpha]^{*} = \{ \pm \alpha^a (1 + p \alpha)^b \mid a, b \in \mathbb{Z} \} $.
\end{Prop}

Note that $\alpha, 1 + p \alpha$ are not necessarily fundamental units in $\Q(\alpha)/\Q$ as we don't claim that the ring of integers of $\Q(\alpha) / \Q$ coincides with $\mathbb{Z}[\alpha]$.

\begin{proof}
We have
\begin{gather}
\alpha (\alpha ^2 + p^2 \alpha) = 1
\\
(1 - p \alpha)(1 + p \alpha) = \alpha^3
\end{gather}
It shows that $\alpha, 1 + p \alpha$ are indeed units.

Let $\alpha_1 < \alpha_2 < \alpha_3 $ be the three different roots of $f$.
We shall compute them approximately.
\begin{gather*}
\alpha_1 = -p^2 + O\left(p^{-4} \right), \\
\alpha_2 = - p^{-1}   + O\left(p^{-4} \right), \\
\alpha_3 = p^{-1} + O\left(p^{-4} \right)
.\end{gather*}

\begin{Rem}
	\label{rem_roots_estimate}
A computation shows that for $p = 100$ we have
\begin{gather*}
\alpha_1 = -9999.99999999000 , \\
\alpha_2 = -0.01000000500000625, \\
\alpha_3 = 0.00999999500000625
.\end{gather*}
It is not important that $p$ is not a prime in this case as the estimate works for any sufficiently large $p$.
\end{Rem}

By the definition of the regulator we have
\begin{multline}
\Reg(\alpha, p\alpha + 1) = 
\begin{vmatrix}
\ln |\alpha_1| 		& \ln |\alpha_3| \\
\ln |p \alpha_1 + 1| 	& \ln |p \alpha_3 + 1|
\end{vmatrix} 
\\
\approx
\begin{vmatrix}
\ln |-p^2| 	& \ln |\frac{1}{p}| \\
\ln |-p^3 + 1| 	& \ln |2|
\end{vmatrix}
=
\ln(p^2)
\ln 2
+
\ln(p^3 - 1) \ln(p)
\\
\approx \ln(p) ( 3 \ln(p) + 2 \ln 2 )
.\end{multline}
Therefore $ \Reg(\alpha, p\alpha + 1) > 0 $ for all prime $p$.

Hence $\alpha, p\alpha + 1$ are independent units. 

\begin{Rem}
For example, for $p = 73$ SageMath computes the regulator to be approximately $\Reg = 61.1719663782187$.

\begin{lstlisting}[style=sagemath, caption=SageMath code]
p = Primes().unrank(20) # p = 73
R.<x> = PolynomialRing(QQ)
P = x^3 + p^2 * x^2  - 1
K.<a> = QQ.extension(P)
print(K.regulator())
\end{lstlisting}
Whereas the above estimate gives $61.1719663782957$.
These numbers coincide up to $10^{-11} $.

Another example, for $p = 547$ SageMath computes the regulator to be approximately $\Reg = 127.978045931846$.

\begin{lstlisting}[style=sagemath, caption=SageMath code]
p = Primes().unrank(100) # p = 547
R.<x> = PolynomialRing(QQ)
P = x^3 + p^2 * x^2  - 1
K.<a> = QQ.extension(P)
print(K.regulator())
\end{lstlisting}
Whereas the above estimate gives $127.97804593184651$.
These numbers coincide up to $10^{-12} $.

\end{Rem}

It is left to prove that they are fundamental units in $\mathbb{Z}[\alpha]$.
By Corollary 5.9 from Conrad it is sufficient to check
\[
\frac{16 \Reg(\alpha, p\alpha + 1) }
{(\ln(\disc_f/4))^2} 
< 2
.\] 
Substituting, we obtain
\[
\frac{16 \Reg(\alpha, 2\alpha + 1) }
{(\ln(\disc_f/4))^2} 
\approx
\frac{16 
\ln(p) ( 3 \ln(p) + 2 \ln 2 )
 }
{(\ln(
(4 p^6 - 27)
/4))^2} 
.\] 
Asymptotically, the latter equals
\[
\stackrel{p \to \infty}{\longrightarrow}
\frac{48
\ln(p)^2
}
{(\ln(
p^6 ))^2} = \frac{4}{3}
.\] 
Therefore it is $< 2$ for big enough $p$, QED.
\end{proof}
\begin{Rem}
In fact, the function
\[
\frac{16 
\ln(p) ( 3 \ln(p) + 2 \ln 2 )
}
{(\ln(
(4 p^6 - 27)
/4))^2} 
\] 
is smaller then $2$
for all $p > 2.6$.
It monotonously decreases to the asymptotic value $\frac{4}{3}$.

The first several values of $g(p) \eqdef 
\frac{16 \Reg(\alpha, 2\alpha + 1) }
{(\ln(\disc_f/4))^2} 
$ are presented in the following
\begin{table}[h!]
\centering
\begin{tabular}{|c|c|}
\hline
$p$ & $g(p)$ \\
\hline
2 & 2.3005757200277737 \\
3 & 0.948774985707828 \\
5 & 1.7162543373046009 \\
7 & 0.824986119068219 \\
11 & 1.5902803808431747 \\
13 & 1.573545207927126 \\
17 & 1.5508005143276846 \\
19 & 1.5425857167054255 \\
23 & 1.529835319790773 \\
29 & 1.5163082966042292 \\
31 & 1.5127547443392193 \\
37 & 1.5039633069049658 \\
41 & 1.4992465878896728 \\
43 & 1.4971456295557026 \\
47 & 1.4933611793019335 \\
53 & 1.4885186045113814 \\
59 & 1.484436992267399 \\
61 & 1.483211643289589 \\
67 & 1.4798674282030506 \\
71 & 1.4778740553564413 \\
73 & 1.4769381932859227 \\
79 & 1.4743421857047407 \\
83 & 1.4727660220987417 \\
89 & 1.4705979224908206 \\
97 & 1.4680152465868979 \\
101 & 1.4668359850881518 \\
103 & 1.466271166102171 \\
107 & 1.4651872592939545 \\
109 & 1.4646667672702098 \\
113 & 1.4636655275160453 \\
127 & 1.4605230591156473 \\
\hline
\end{tabular}
\caption{Values of $p$ and $g(p)$}
\label{table:gp}
\end{table}
\newpage
which we computed using SageMath
\begin{lstlisting}[style=sagemath, caption=SageMath code]
for i, p in enumerate(Primes()):
    if i > 30:
        break
    R.<x> = PolynomialRing(QQ)
    P = x^3 + p^2 * x^2 - 1
    K.<a> = QQ.extension(P)
    rg = K.regulator()
    print(p, float(16 * rg * ln( p^6 - 27/4 )^(-2)))
\end{lstlisting}
\end{Rem}

\begin{Rem}
We just proved that the regulator is approximately 
\[
\ln(p) ( 3 \ln(p) + 2 \ln 2 )
,\] 
which is close to $3 \ln^2 p$ we wanted from the beginning.
\end{Rem}

\section{Centralizers}
\begin{Prop}
The centralizer of $\tilde A$ in $\SL$
is generated by $A, p A+\Id$.
$$Z_{\SL}(\tilde A) = \{ A^a (p A + \Id)^b \mid a, b \in \mathbb{Z} \} .$$ 
\end{Prop}
\begin{proof}
Since $\tilde A$ is regular, its centralizer in $ \Mat $ is $\C \left< \Id, A, A^2 \right> $.
Now, 
\[
\C \left< \Id, A, A^2 \right> \cap \SL 
\subset \mathbb{Z} 
\left< \Id, A, A^2 \right> = \mathbb{Z}[A]
.\] 
Indeed,
\begin{equation}
\Id = 
\begin{pmatrix}
1 & 0 & 0 \\
0 & 1 & 0 \\
0 & 0 & 1
\end{pmatrix}, 
A = 
\begin{pmatrix}
0 & 0 & 1 \\
1 & 0 & 0 \\
0 & 1 & -p^2
\end{pmatrix}, 
A^2 = 
\begin{pmatrix}
0 & 1 & -p^2 \\
0 & 0 & 1 \\
1 & -p^2 & p^4
\end{pmatrix}, 
\end{equation}
Considering the first matrix column we see that if a complex combination has integer coefficients, it is in fact integer combination.

Moreover, 
the centralizer of $\tilde A$ is a group, therefore it lies inside the multiplicative group of $\mathbb{Z}[A]$
\[
Z_{\SL}(\tilde A)  \subset 
\C \left< \Id, A, A^2 \right> \cap \SL 
\subset \mathbb{Z}[A]^*
.\] 
There is an isomorphism of $\mathbb{Z}$-algebras $\mathbb{Z}[A] \simeq \mathbb{Z}[x]/(f(x)) = \mathbb{Z}[\alpha]$.
Applying Proposition \ref{prop_multiplicative_group_structure} end the proof
\[
Z_{\SL}(\tilde A) 
\subset \mathbb{Z}[A]^* =\{ \pm A^a (p A + \Id)^b \mid a, b \in \mathbb{Z} \}.\] 
\end{proof}

We are to study the centralizer of $\tilde A$ in $\SLp$.
\[
Z_{\SLp}(\tilde A) \subset 
Z_{\SL}(\tilde A) \cong \mathbb{Z}^2
.\] 

\begin{Prop}
	The index of the one centralizer inside the other \emph{does not depend on $p$} and equals $3$.
\[
	[Z_{\SL}(\tilde A) : Z_{\SLp}(\tilde A) ] = 3
.\] 
\end{Prop}
\begin{proof}
As $A \pmod p$ is equivalent to the cyclic permutation matrix, it follows that $A^3 \in \SLp$.

Thus the smaller centralizer is generated by $A^3, \tilde A$
\[
Z_{\SLp}(\tilde A)
= \{ A^{3a} (p A + \Id)^b \mid a, b \in \mathbb{Z} \}
,\] 
and it is clear that the index is $3$.
\end{proof}

\section{Computing (co)homology}
Christophe Soulé \cite{soule1978}
computed the cohomology of $\SL$
to be
\begin{table}[h!]
	\centering
\begin{tabular}{|c|c|c|c|c|c|c}
	\hline
	$n$        & $0$                               & $1$                               & $2$                               & $3$                               & $4$  & \ldots\\
\hline 
	$H^n(\SL, \mathbb{Z})$ & $\mathbb{Z}$ & $0$ & $0$ & $(\mathbb{Z}/2\mathbb{Z})^{2} $ & $(\mathbb{Z}/3\mathbb{Z})^{2} \oplus (\mathbb{Z}/4\mathbb{Z})^{2} $ & \ldots \\
\hline
\end{tabular}.
\end{table}

The universal coefficient formula gives us the homology.
\begin{table}[h!]
	\centering
\begin{tabular}{|c|c|c|c|c|c}
	\hline
	$n$        & $0$                               & $1$                               & $2$                               & $3$                               & \ldots\\
\hline 
	$H_n(\SL, \mathbb{Z})$ & $\mathbb{Z}$ & $0$ &  $(\mathbb{Z}/2\mathbb{Z})^{2} $ & $(\mathbb{Z}/3\mathbb{Z})^{2} \oplus (\mathbb{Z}/4\mathbb{Z})^{2} $ & \ldots \\
\hline
\end{tabular}
\end{table}

Recall the matrices 
\[
A = 
\begin{pmatrix}
0 & 0 & 1 \\
1 & 0 & 0 \\
0 & 1 & -p^2
\end{pmatrix}
,
B = \Id + p A =
\begin{pmatrix}
1 & 0 & p \\
p & 1 & 0 \\
0 & p & 1 - p^3
\end{pmatrix}
.\] 
We denote $\tilde A$ by $B$ for convenience.

As $A, B $ commute they define a map
\[
f: \mathbb{Z}^2 \to \SL, 
\] 
\[
(k, l) \mapsto A^k B^l
.\] 

\begin{Thm}
	\label{thm_push_homology_vanishes}
The pushforward map in the second homology 
\[
f_*:
H_2(\mathbb{Z}^2, \mathbb{Z}) \cong \mathbb{Z} \to 
H_2(\SL, \mathbb{Z}) \cong (\mathbb{Z}/2\mathbb{Z})^{2} 
\] 
is zero.
\end{Thm}
The goal of this section is to prove this theorem.

The first observation is that one can pass from the homology to the cohomology with coefficients $\mathbb{Z}/ 2 \mathbb{Z}$.

\begin{Lem}
	\label{lemma_push_vanishes_iff_pull_vanishes}
The pushforward
\[
f_*:
H_2(\mathbb{Z}^2, \mathbb{Z}) \cong \mathbb{Z} \to 
H_2(\SL, \mathbb{Z}) \cong (\mathbb{Z}/2\mathbb{Z})^{2} 
\] 
is zero 
iff
the pullback
\[
f^*:
H^2(\SL,  \mathbb{Z}/2 \mathbb{Z}) \cong (\mathbb{Z}/2\mathbb{Z})^{2} \to
H^2(\mathbb{Z}^2, \mathbb{Z}/2 \mathbb{Z}) \cong \mathbb{Z}/2 \mathbb{Z}
\] 
is zero.
\end{Lem}
\begin{proof}
As the first homology 
\[
H_1(\mathbb{Z}^2, \mathbb{Z}) \cong \mathbb{Z}^2, 
H_1(\SL, \mathbb{Z}) \cong 0
\] 
are torsion-free, the universal coefficien formula gives perfect $\mathbb{Z}/2\mathbb{Z}$-pairings between homology and cohomology
\[
H^2(\mathbb{Z}^2, \mathbb{Z}/2 \mathbb{Z}) = \Hom(H_2(\mathbb{Z}^2, \mathbb{Z}), \mathbb{Z}/2 \mathbb{Z})
,\] 
\[
H^2(\SL, \mathbb{Z}/2 \mathbb{Z}) = \Hom(H_2(\SL, \mathbb{Z}), \mathbb{Z}/2 \mathbb{Z})
.\] 
Therefore the pullback in cohomology is the transposed map of the pushforward in homology and one is zero iff the other is zero.
\end{proof}

Next, let us find a nicer description for $H^2(\SL, \mathbb{Z}/2 \mathbb{Z})$.

\begin{Cla}
\begin{itemize}
	\item The second cohomology $H^2(G, M)$ classify the extensions $0 \to M \to \tilde G \to G \to 0$;
	\item In particular, $H^2(G, \mathbb{Z}/2\mathbb{Z})$ correspond to the extensions of degree $2$;
\item Given an extension $0 \to M \to \tilde G \to G \to 0$ and a principle $G$-bundle over $X$ we get its class in $H^2(X, M)$ via the boundary (in other words, Bockstein) homomorphism $H^1(X, G) \to H^2(X, M)$.  In details, the Bockstein goes by taking the $1$-cocycle defining the $G-$bundle to $\tilde G$ and apply the Cech differential;
	\item $H^2( \mathbf{SL}_3(\mathbb{R}), \mathbb{Z}/ 2 \mathbb{Z}) \cong
	H^2( \mathbf{SL}_3(\mathbb{F}_2), \mathbb{Z}/ 2 \mathbb{Z}) \cong \mathbb{Z}/2\mathbb{Z}$;
\item The corresponding unique extensions of degree two are $\widetilde{\mathbf{SL}_3}(\mathbb{R})$ and $\mathbf{SL}_2(\mathbb{F}_7)$.
	We make use of the exceptional isomorphism $\mathbf{PSL}_2(\mathbb{F}_7) \cong \mathbf{SL}_3(\mathbb{F}_2)$, see, for example, \cite{merzon_psl_2_7_sl_3_2}.
\item If $E$ is a principle $\mathbf{SL}_3(\mathbb{R})$-bundle on $X$ the corresponding class in $H^2(X, \mathbb{Z}/ 2 \mathbb{Z})$ is called $w_2(E)$, the second Stiefel-Whitney class;
\item If $E$ is a principle $\mathbf{SL}_3(\mathbb{F}_2)$-bundle on $X$ let us denote the corresponding class in $H^2(X, \mathbb{Z}/ 2 \mathbb{Z})$ by $u_2(E)$.
\end{itemize}
\end{Cla}
This is all very well known.
For the cohomology of $\mathbf{SL}_3(\mathbb{F}_2)$ see, for example, \cite{adem2006lecturescohomologyfinitegroups}.

Now, consider the natural mappings
\[
i: \SL \embeds {\mathbf{SL}_3(\mathbb{R})}
, \] 
\[
\pi: \SL \embeds {\mathbf{SL}_3(\mathbb{F}_2)}
.\] 

\begin{Prop}
The pullback 
\[
i^* \oplus \pi^*: 
H^2(B\mathbf{SL}_3(\mathbb{R}), \mathbb{Z}/ 2 \mathbb{Z}) \oplus
H^2(B\mathbf{SL}_3(\mathbb{F}_2), \mathbb{Z}/ 2 \mathbb{Z}) \cong (\mathbb{Z} / 2 \mathbb{Z})^2 \to 
H^2(\SL, \mathbb{Z}/ 2 \mathbb{Z}) \cong (\mathbb{Z}/2\mathbb{Z})^{2}
\] 
is an isomorphism.
\end{Prop}

\begin{proof}
It is enough to show that a principal $\SL$ bundle $E$ on $\mathbb{T}^2$ can have an arbitrary combination of $(w_2(E), u_2(E)) \in (\mathbb{Z}/ 2\mathbb{Z})^2$.

Plan: we shall construct principal bundles $E_1, E_2$ on $\mathbb{T}^2$ with $(w_2(E_1), u_2(E_1)) = (1, 0)$ and $u_2(E_2) = 1$.
This will prove the proposition.

The first bundle $E_1$ is defined via the monodromy matrices $C_1, C_2$
\[
C_1 = 
\begin{pmatrix}
-1 & 0 & 0 \\
0 & -1 & 0 \\
0 & 0 & 1
\end{pmatrix}
,
C_2 =
\begin{pmatrix}
1 & 0 & 0 \\
0 & -1 & 0 \\
0 & 0 & -1
\end{pmatrix} \in \SL
.\] 
By the splitting principle we have
\[
w(E_1)
= (1 + x) (1 + x + y) (1 + y) 
= 1 + xy
,\] 
where $x, y$ denote the generators of $H^1(\mathbb{T}^2, \mathbb{Z}/ 2 \mathbb{Z})$.
Therefore $w_2(E_1) = 1$.
As modulo $2$ the matrices $C_1, C_2$ are the identity the bundle $E_1$ mod $2$ is trivial; thus $u_2(E_1) = 0$.

Define $D_1' = 
\left( \begin{smallmatrix}0&6\\1&0\end{smallmatrix}\right) 
, D_2' =
\left( \begin{smallmatrix}2&3\\3&5\end{smallmatrix}\right)
\in \mathbf{SL}_2(\mathbb{F}_7)$.
Project the matrices to 
$\in \mathbf{PSL}_2(\mathbb{F}_7) \cong \in \mathbf{SL}_3(\mathbb{F}_2) $.
Pick a pair of commuting lifts $D_1, D_2 \in \mathbf{SL}_3(\mathbb{Z})$.
One can take them to be 
\begin{equation}
D_1 = 
\begin{pmatrix}
1 & 1 & 0 \\
0 & 1 & 0 \\
0 & 1 & 1
\end{pmatrix}, 
D_2 =
\begin{pmatrix}
1 & 0 & 0 \\
0 & 1 & 0 \\
0 & 1 & 1
\end{pmatrix}
.\end{equation}

Define the bundle $E_2$ corresponding to the monodromy matrices $D_1, D_2$. 
We claim that $u_2(E_2) = 1$.

Indeed, we shall compute the Bockstein morphism $H^1(\mathbb{T}^2, \mathbf{SL}_3(\mathbb{F}_2) \to H^2(\mathbb{T}^2, \mathbb{Z}/ 2 \mathbb{Z})$ by lifting the cocycles.

One checks that $D_1' D_2' = - D_2' D_1'$; thus the image of the Bockstein map is nontrivial.

Thus we have showed that $E_1, E_2$ have the right characteristic classes and we are done.
\end{proof}
\begin{proof}[Proof of the Theorem \ref{thm_push_homology_vanishes}]
In view of the Lemma \ref{lemma_push_vanishes_iff_pull_vanishes} we shall show that the pullback map in second $\mathbb{Z}/ 2 \mathbb{Z}$-cohomology vanishes.

Consider the principal $\SL$-bundle $E$ on $\mathbb{T}^2$ associated with $A, B$.
We should show that $w_2(E) = 0$ and $u_2(E) = 0$.

By the splitting principle
using that the roots of the characteristic polynomial 
\ref{rem_roots_estimate} 
satisfy \[
\alpha_1, 1 + p \alpha_1 < 0, 
\alpha_2, 1 + p \alpha_2 < 0, 
\alpha_3, 1 + p \alpha_3 > 0 
\] 
we get the Stiefel-Whitney class to be
\[
w(E) = w(E_1) w(E_2) w (E_3)
= (1 + x + y) (1 + x + y) (1) 
= 1
,\]
and thus $w_2(E) = 0$ indeed does vanish.

Modulo $2$ one easily spots that $B = A^5 \pmod{2}$, thus their lifts to $\mathbf{SL}_2(\mathbf{F}_7)$ do commute and the image of the Bockstain morphism vanishes; thus $u_2(E) = 0$.
% the explanation is that we should consider SL_3(F_2) in the automorphisms of F_8 = F_2[t]/t^3 + t^2 + 1 as a linear space over F_2; then A is multiplication by t and B is a multiplication by 1 + t = t^5. 
\end{proof}



\newpage
\section{From $\SL$ to $\SLn$}

Consider
\begin{itemize}
\item $p > 2$ a prime number;
\item $\SLn$ the $n$-th special linear group over $\mathbb{Z}$;
\item $\SLp$ the $p$th congruence subgroup of $\SLn$;
\item $Z_G(x)$ the centralizer of $x \in G$;
\item $ f(t) \eqdef t^n - (1 + pt) (1 + 2pt) \ldots (1 + (n - 1) p t)$;
\item If $C$ is a matrix, $\chi_C(\lambda) \assign \det(\lambda \Id - C)$ is the characteristics polynomial.
\item Let $A$ be the irreducible operator with $\chi_A(t) = f(t)$; 

For example, when $n = 4$ $A$ takes a form
\begin{equation}
A = 
\begin{pmatrix}
0 & 0 & 0 & 1 \\
1 & 0 & 0 & 6 p \\
0 & 1 & 0 & 11 p^2 \\
0 & 0 & 1 & 6 p^3
\end{pmatrix}
\in {\mathbf{SL}_4(\mathbb{Z})};        % SL_3(Z)
\end{equation}
\item 
Define $\tilde A \eqdef \Id + p A$.

For example, when $n = 4$ $\tilde A$ takes a form
\begin{equation}
\tilde A = \Id + p A =
\begin{pmatrix}
1 & 0 & 0 & p \\
p & 1 & 0 & 6 p^2 \\
0 & p & 1 & 11 p^3 \\
0 & 0 & p & 1 + 6 p^4
\end{pmatrix}
\in \SLp.
\end{equation}
\end{itemize}

\section{Intro2}
In this note we establish the following results:
\begin{enumerate}
\item
The regularor of $\tilde A$
grows as $O (\ln^{n-1} (p)) $; 

\item
If $n$ is prime,
the index of the centralizers
$ [Z_{\SLn}(\tilde A) : Z_{\SLp}(\tilde A)] < C$
is bounded by some constant $C$ and does not depend on $p$.
\end{enumerate}
This might be interesting.

\section{Regulators}
We mimic the proofs from a Keigh Conrad's write-up on the Dirichlet unit theorem \cite{conraddirichlet}.

There are more general estimates due to Duke \cite{duke2003extreme} (see Section 3 Proposition 1).


Let $K \eqdef \Q(\alpha)$ be the number field obtain by attaching the root of $f(t)$ to $\Q$.
\begin{Prop}
$K / \Q$ is a totally real number field of degree $n$.
\end{Prop}

\begin{proof}
First, $f(t)$ is irreducible as it has no rational roots: 
\begin{gather}
f(1) = 1 - (1 + p) (1 + 2p) \ldots (1 + (n - 1) p) \ne 0,\\
f(-1) = (-1)^n - (1 - p) (1 - 2p) \ldots (1 - (n - 1) p ) \ne 0.
\end{gather}

Let's compute the roots of $f(t)$ approximately and show they are real.

Notice that for $|t| \ll 1$ all but one roots of $f(t)$ are close to the roots of 
\[
(1 + pt) (1 + 2pt) \ldots (1 + (n - 1) p t)
,\] 
them being $p^{-1} , \frac{1}{2} p^{-1} , \ldots, \frac{1}{(n-1) } p^{-1} $.

As all but one roots are real, the remaining one is also real.
\end{proof}

The next goal is to tackle the unit group $\O_K^*$ of $K$.

The Dirichlet unit theorem implies that $\O_K^*$ is an abelian group of rank $n-1$ (there might be some finite torsion of roots of unity.  It is bounded universally in $p$).
% depending only on $n$
% as the degree of the roots of unity is the totient function which grows as $n \to \infty$.
A basis of $\O_K^*$ modulo roots of unity is called a system of fundamental units.
Finding a system of fundamental units in general is a complicated task.

Instead, we shall find a finite index subgroup $U \subset \O_K^*$. 
% \red{maybe order $\O \subset \O_K$ instead?}
Then we shall bound above the regulator of $K$ with the regulator of $U$.

Moreover, using a Silverman's lower bound on the regulator of $K$ we bound above the index $[\O_K^*: U] < C_1$ with $C_1$ independent on $p$.
% \red{maybe say uniformly in $p$ instead?}


Define a subgroup 
\[
	U = (1 + p \alpha)^{\mathbb{Z}} \ldots (1 + (n-1) p \alpha)^{\mathbb{Z}} \subset \O_K^*
.\] 
\begin{Prop}
$U \subset \O_K^*$ is a finite index subgroup.
The regulator $\Reg_U$ is bounded above by $C_2 \log(p)^{n-1}$ where $C_2$ is a constant independent on $p$.
\end{Prop}

\begin{proof}
We have
\begin{gather}
\alpha^n - (1 + p\alpha) (1 + 2p\alpha) \ldots (1 + (n - 1) p\alpha) + 1 = 1  ,
\\
(1 + p\alpha) (1 + 2p\alpha) \ldots (1 + (n - 1) p\alpha) = \alpha^n.
\end{gather}
It shows that $1 + p \alpha, \ldots, 1 + (n - 1) p \alpha$ are indeed units.

Let $\alpha_1 < \ldots < \alpha_{n-1} < 0 $ be the negative roots of $f$.
We shall compute them approximately.
\begin{Lem}
\[
\alpha_k = \frac{-1}{k} p^{-1} + \frac{(-1)^{k + 1} }{ k\cdot  k! (n - k + 1)!} p^{-n-1}  + O(p^{-2 n - 1})
.\] 
\end{Lem}
\begin{proof}
In the first approximation $\alpha_k = \frac{-1}{k} p^{-1} + o(p^{-1})$.
Let's compute the residue term more persisely. 
Let $\alpha_k = \frac{-1}{k} p^{-1} ( 1 + \epsilon )$.
Substituting to $f(t)$, we get
\begin{gather*}
(1 + p\alpha_k) (1 + 2p\alpha_k) \ldots (1 + (n - 1) p\alpha_k) = \alpha_k^n, \\
- (- k p\alpha_k - k) (- 2 k p\alpha_k - k)  \ldots (- (n - 1) k p\alpha_k - k)  = (- k \alpha_k) ^n, \\
- (1 - \epsilon - k) (2 - 2 \epsilon - k)  \ldots ( (n-1) - (n-1) \epsilon - k)  = p^{-n} (1 + \epsilon)^n, \\
- (1 - k) (2 - k) \ldots (- 1) k \epsilon 1 \ldots ( (n-1) - k) + O ( \epsilon^2)
= p^{-n} + O( \epsilon p^{-n} ) , \\
(-1)^k k! ( n - k - 1)! \epsilon + O ( \epsilon^2)
= p^{-n} + O( \epsilon p^{-n} ) , \\
\epsilon = \frac{ (-1)^k p^{-n} }{ k! ( n - k - 1)! }  + O( \epsilon p^{-n} ) + O ( \epsilon^2), \\
\epsilon = \frac{ (-1)^k p^{-n} }{ k! ( n - k - 1)! }  + O( p^{-2n} ).
\end{gather*}
\end{proof}

By the definition of the regulator we have
\begin{multline}
\Reg_U = 
\left|
\begin{array}{cccc}
\ln |1 + p \alpha_1| 	& \ln |1 + p \alpha_2| & \ldots & \ln |1 + p \alpha_{n-1}| \\
\ln |1 + 2 p \alpha_1| 	& \ln |1 + 2 p \alpha_2| & \ldots &  \ln |1 + 2 p \alpha_{n-1}| \\
\vdots 	& \vdots & \ddots &  \vdots \\
\ln |1 + (n-1) p \alpha_1| 	& \ln |1 + (n-1) p \alpha_2| & \ldots &  \ln |1 + (n-1) p \alpha_{n-1}| 
\end{array} 
\right|
\\
\approx
\left|
\begin{array}{cccc}
\ln \left( \frac{ p^{-n}}{n!}  \right)  	& \ln \left( 1 - \frac{1}{2} \right)  & \ldots &  \ln \left( 1 - \frac{1}{n - 1} \right)  \\
\ln |1 - 2| 	& \ln \left( \frac{p^{-n} }{2\cdot  2! (n-1)!} \right)  & \ldots &  \ln \left( 1 - \frac{2}{n-1} \right)  \\
\vdots 	& \vdots & \ddots &  \vdots \\
\ln |1 - (n-1) | 	& \ln |1 - \frac{n-1}2| & \ldots &  \ln \left( \frac{p^{-n}}{(n-1)\cdot (n-1)! 2!} \right) 
\end{array} 
\right| \\
\approx n^{n-1} \log(p)^{n-1} 
.\end{multline}
\end{proof}
From the proof of the Proposition above one finds that the regulator of $U$ $ \Reg_U > 0 $ is positive for big enough prime $p$.
Hence the generators $(1 + p \alpha), \ldots, (1 + (n-1) p)$ of $U$ are multiplicatively independent, and $U \subset \O_K^*$ is a finite index subgroup.

We shall bound the index $[\O_K^* : U]$ from above via the following
\begin{Thm}[Silverman, \cite{silverman1984inequality}]
Let $K$ be a number field of degree $d$ with regulator $R$, and absolute discriminant $D_K$. 
Let $r(K)$ be the rank of the unit group in $K$, and let $p(K)$ be the maximum of $r(k)$ as $k$ ranges over proper subfields of $K$. 
We prove 
\[
R_K > c_d (\log \gamma_d  D_K )^{r(K) - p(K)} 
\] 
for constants $c_d, \gamma_d$ depending only on $d$. 
\end{Thm}
\red{It seems it's actually due to 
Robert Remak in
"Über Größenbeziehungen zwischen Diskriminante
und Regulator eines algebraischen Zahlkörpers" }

\begin{Prop}
Let $n$ be a prime.
Then the index $[\O_K^* : U] < C_3$ is bounded from above by a constant $C_3$ independent on $p$.
\end{Prop}
\begin{proof}
The discriminant of $K$ is the discriminant of the polynomial $f(t)$, which is approximately
\[
	\disc_K = \disc_f = O( p^{2 (n - 1)^2} p^{-(n - 1) (n - 2)} ) = O ( p^{n (n - 1) } )
.\] 
More precisely,
\[
	\disc_K = \disc_f \approx 
	1!^2 \ldots (n-1)!^2 p^{n (n - 1) }
.\] 

Given that $n$ is prime and so there are no nontrivial subfield of $K$ the Silverman's theorem implies 
$$\Reg_K > C_4 (\log \disc_K)^{n-1} \approx n^{n-1} (n-1)^{n-1} C_4 \log(p)^{n-1}  > C_5\log(p)^{n-1}.$$

It is easy to see that 
\[
\Reg_U = [\O_K^* : U] \Reg_K
.\] 
Thus we have
\[
 C_2 \log(p)^{n-1} > \Reg_U = 
[\O_K^* : U] \Reg_K >
[\O_K^* : U] C_5 \log(p)^{n-1}
.\] 
It follows that the index is bounded from above
\[
[\O_K^* : U] < \frac{C_2}{C_5}
,\] 
as desired.
\end{proof}


\section{Centralizers}
\begin{Prop}
Let $n$ be a prime.  
Then the index of the centralizers
\[
	[Z_{\SLn}(\tilde A) : Z_{\SLp}(\tilde A)] < C_3
\] 
is bounded from above by a constant independent of $p$.
\end{Prop}
\begin{proof}
	We shall exploit the properties of the previously introduced number field $K = \mathbb{Q}(\alpha)$, where $\alpha$ is a root of $f(t) = t^n - (1 + pt) \ldots (1 + (n-1) pt)$.

Since $A$ is regular, 
% over $\mathbb{Z}$
there is an isomorphism of $\mathbb{Z}$-algebras 

\[
\Phi: \mathbb{Z}[A] 
\mathop{\longrightarrow}^{\sim}
\mathbb{Z}[\alpha], A \mapsto \alpha
.\] 

It identifies the stabilizer of $\tilde A$ in $\SLn$ with the multiplicative group of the order $\mathbb{Z}[\alpha] \subset \O_K$.
\[
Z_{\SLn}(\tilde A) 
= Z_{\SLn}(A) 
= \mathbb{Z}[A]^*
\mathop{\simeq }^{\Phi} 
\mathbb{Z}[\alpha]^*
.\] 

The isomorphism $\Phi$ identifies
the subgroup $U \subset \mathbb{Z}[\alpha]^*$ with
$$
\Phi^{-1} (U) = \tilde A^{\mathbb{Z}} (2 \tilde A - \Id )^{\mathbb{Z}} \ldots   ((n-1) \tilde A - (n-2) \Id )^{\mathbb{Z}} \subset Z_{\SLp}(\tilde A) 
.$$

Thus, there is a chain of inclusions
\[
\Phi^{-1} (U) \subset Z_{\SLp}(\tilde A) 
\subset Z_{\SLn}(\tilde A) = \Phi^{-1} \mathbb{Z}[\alpha]^*
.\] 
Therefore, there is a bound
\[
[Z_{\SLn}(\tilde A) : Z_{\SLp}(\tilde A) ]
\le
[\mathbb{Z}[\alpha]^*, U]
\le 
[\O_K^*, U] < C_3
.\] 
\end{proof}


\bibliography{mybib}
\bibliographystyle{alpha}

\end{document}
