\documentclass[a4paper]{article}

\usepackage[utf8]{inputenc}
\usepackage[T1]{fontenc}
\usepackage{textcomp}
\usepackage[english]{babel}
\usepackage{amsmath, amssymb, amsthm, mathtools}
\usepackage{hyperref} % clickable links in the document
\usepackage{url} % clickable links in the document
\usepackage{enumitem} % (a), (b), etc labels

\usepackage{array}

\usepackage{xcolor} % support of \red text
\newcommand{\red}{\textcolor{red}} % support of \red text

\usepackage{listings}

% Define SageMath style for highlighting
\lstdefinestyle{sagemath}{
    language=Python, % SageMath is based on Python
    basicstyle=\ttfamily\footnotesize,
    commentstyle=\color{gray},
    keywordstyle=\color{blue},
    stringstyle=\color{red},
    frame=single,
    breaklines=true,
    showstringspaces=false,
    tabsize=4,
    captionpos=b,
    morekeywords={R, PolynomialRing, QQ} % Add SageMath specific keywords here
}

\newtheorem{Thm}{Theorem}[section]
\newtheorem{Lem}[Thm]{Lemma}
\newtheorem{Def}[Thm]{Definition}
\newtheorem{Cor}[Thm]{Corollary}
\newtheorem{Prop}[Thm]{Proposition}
\newtheorem{Rem}[Thm]{Remark}
\newtheorem{Ex}[Thm]{Example}
\newtheorem{Cla}[Thm]{Claim}
\newtheorem{Exer}[Thm]{Exercise}
\newtheorem{Prob}[Thm]{Problem}

\newcommand{\embeds}{\hookrightarrow}
\newcommand{\projects}{\twoheadrightarrow}
\newcommand{\eqdef}{\stackrel{\mathrm{def}}{=}}

\DeclareMathOperator{\assign}{\coloneqq}        % Assingment operator
\newcommand{\R}{\mathbb{R}}        % Real
\newcommand{\Q}{\mathbb{Q}}        % rationals
\newcommand{\C}{\mathbb{C}}        % Complex
\renewcommand{\P}{\mathbb{P}}        % Projective
\newcommand{\A}{\mathbb{A}}        % Affine
\renewcommand{\O}{\mathcal{O}}        % Structure sheaf
\newcommand{\Id}{\mathbf{Id}}        % Identity
\newcommand{\SL}{\mathbf{SL}_3(\mathbb{Z})}        % SL_3(Z)
\newcommand{\GL}{\mathbf{GL}_3} % GL_3
\newcommand{\PGL}{\mathbf{GL}_3} % PGL_3
\newcommand{\Mat}{\mathbf{Mat}_3(\mathbb{C})}        % Mat_3( C )
\newcommand{\MatZ}{\mathbf{Mat}_3(\mathbb{Z})}        % Mat_3( ZZ )
\newcommand{\SLp}{\Gamma_p}        % a congruence subgroup
\newcommand{\phip}{\phi_p}        % the golden ratio mod(p)
\newcommand{\hence}{\Rightarrow}        % hence
\newcommand{\minpoly}{\mathrm{minpoly}}        % minimal polynomial

\DeclareMathOperator{\Spec}{Spec}        % Spec
\DeclareMathOperator{\Proj}{Proj}        % Proj
\DeclareMathOperator{\Reg}{Reg}        % Regulator
\DeclareMathOperator{\disc}{disc}        % discriminant


\begin{document}
\title{A sample computation}	
\author{Altan Erdnigor}
\maketitle

\tableofcontents

\section{Notation}
\begin{itemize}
\item $p > 2$ a prime number.
\item $\SL$ the special linear group over $\mathbb{Z}$.
\item $\SLp$ the $p$th congruence subgroup of $\SL$.
\item $Z_G(x)$ the centralizer of $x \in G$.
\item 
\begin{equation}
\label{matrix_small}
A = 
\begin{pmatrix}
0 & 0 & 1 \\
1 & 0 & 0 \\
0 & 1 & -p^2
\end{pmatrix}
\in \SL
\end{equation}
\item 
\begin{equation}
\label{matrix_big}
\tilde A = \Id + p A =
\begin{pmatrix}
1 & 0 & p \\
p & 1 & 0 \\
0 & p & 1 - p^3
\end{pmatrix}
\in \SLp
\end{equation}

\item If $C$ is a matrix, $\chi_C(\lambda) \assign \det(\lambda \Id - C)$ is the characteristics polynomial.
\item $ f(t) \assign \chi_A(t) = t^3 + p^2 t^2 - 1$.

\end{itemize}

\section{Intro}
In this note we establish the following results:
\begin{enumerate}
\item
The regularor of $\tilde A$
grows as $\approx 3 \ln^2(p)$. 

That is, $\Reg ( \Q(\alpha)/\Q ) \approx 3 \ln^2(p) $ for $\alpha$ a root of $\chi_{\tilde A}$.

\item
The index of the centralizers
$ [Z_{\SL}(\tilde A) : Z_{\SLp}(\tilde A)] = 3$
does not depend on $p$.
\end{enumerate}
This might be interesting.

\section{Regulators}
We mimic the proofs from a Keigh Conrad's write-up on the Dirichlet unit theorem \cite{conraddirichlet}.


Let $\Q(\alpha)$ be the number field obtain by attaching the root of $f(t)$ to $\Q$.
\begin{Prop}
$\Q(\alpha) / \Q$ is a totally real number field of degree $3$.
\end{Prop}

\begin{proof}
First, $f(t)$ is irreducible as it has no rational roots: $f(1) = p^2 \ne 0, f(-1) = p^2 - 2 \ne 0 $.

The discriminant of $f(t)$ equals
$$\disc_f(p) = 4 p^6 - 27 .$$

The discriminant is always positive $\disc_f(p) > 0$, thus the number field is totally real.
\end{proof}


\begin{Prop}
\label{prop_multiplicative_group_structure}
$\mathbb{Z}[\alpha]^{*} = \{ \pm \alpha^a (1 + p \alpha)^b \mid a, b \in \mathbb{Z} \} $.
\end{Prop}

Note that $\alpha, 1 + p \alpha$ are not necessarily fundamental units in $\Q(\alpha)/\Q$ as we don't claim that the ring of integers of $\Q(\alpha) / \Q$ coincides with $\mathbb{Z}[\alpha]$.

\begin{proof}
We have
\begin{gather}
\alpha (\alpha ^2 + p^2 \alpha) = 1
\\
(1 - p \alpha)(1 + p \alpha) = \alpha^3
\end{gather}
It shows that $\alpha, 1 + p \alpha$ are indeed units.

Let $\alpha_1 < \alpha_2 < \alpha_3 $ be the three different roots of $f$.
We shall compute them approximately.
\begin{gather*}
\alpha_1 = -p^2 + O\left(p^{-4} \right), \\
\alpha_2 = - p^{-1}   + O\left(p^{-4} \right), \\
\alpha_3 = p^{-1} + O\left(p^{-4} \right)
.\end{gather*}

\begin{Rem}
A computation shows that for $p = 100$ we have
\begin{gather*}
\alpha_1 = -9999.99999999000 , \\
\alpha_2 = -0.01000000500000625, \\
\alpha_3 = 0.00999999500000625
.\end{gather*}
It is not important that $p$ is not a prime in this case as the estimate works for any sufficiently large $p$.
\end{Rem}

By the definition of the regulator we have
\begin{multline}
\Reg(\alpha, p\alpha + 1) = 
\begin{vmatrix}
\ln |\alpha_1| 		& \ln |\alpha_3| \\
\ln |p \alpha_1 + 1| 	& \ln |p \alpha_3 + 1|
\end{vmatrix} 
\\
\approx
\begin{vmatrix}
\ln |-p^2| 	& \ln |\frac{1}{p}| \\
\ln |-p^3 + 1| 	& \ln |2|
\end{vmatrix}
=
\ln(p^2)
\ln 2
+
\ln(p^3 - 1) \ln(p)
\\
\approx \ln(p) ( 3 \ln(p) + 2 \ln 2 )
.\end{multline}
Therefore $ \Reg(\alpha, p\alpha + 1) > 0 $ for all prime $p$.

Hence $\alpha, p\alpha + 1$ are independent units. 

\begin{Rem}
For example, for $p = 73$ SageMath computes the regulator to be approximately $\Reg = 61.1719663782187$.

\begin{lstlisting}[style=sagemath, caption=SageMath code]
p = Primes().unrank(20) # p = 73
R.<x> = PolynomialRing(QQ)
P = x^3 + p^2 * x^2  - 1
K.<a> = QQ.extension(P)
print(K.regulator())
\end{lstlisting}
Whereas the above estimate gives $61.1719663782957$.
These numbers coincide up to $10^{-11} $.

Another example, for $p = 547$ SageMath computes the regulator to be approximately $\Reg = 127.978045931846$.

\begin{lstlisting}[style=sagemath, caption=SageMath code]
p = Primes().unrank(100) # p = 547
R.<x> = PolynomialRing(QQ)
P = x^3 + p^2 * x^2  - 1
K.<a> = QQ.extension(P)
print(K.regulator())
\end{lstlisting}
Whereas the above estimate gives $127.97804593184651$.
These numbers coincide up to $10^{-12} $.

\end{Rem}

It is left to prove that they are fundamental units in $\mathbb{Z}[\alpha]$.
By Corollary 5.9 from Conrad it is sufficient to check
\[
\frac{16 \Reg(\alpha, p\alpha + 1) }
{(\ln(\disc_f/4))^2} 
< 2
.\] 
Substituting, we obtain
\[
\frac{16 \Reg(\alpha, 2\alpha + 1) }
{(\ln(\disc_f/4))^2} 
\approx
\frac{16 
\ln(p) ( 3 \ln(p) + 2 \ln 2 )
 }
{(\ln(
(4 p^6 - 27)
/4))^2} 
.\] 
Asymptotically, the latter equals
\[
\stackrel{p \to \infty}{\longrightarrow}
\frac{48
\ln(p)^2
}
{(\ln(
p^6 ))^2} = \frac{4}{3}
.\] 
Therefore it is $< 2$ for big enough $p$, QED.
\end{proof}
\begin{Rem}
In fact, the function
\[
\frac{16 
\ln(p) ( 3 \ln(p) + 2 \ln 2 )
}
{(\ln(
(4 p^6 - 27)
/4))^2} 
\] 
is smaller then $2$
for all $p > 2.6$.
It monotonously decreases to the asymptotic value $\frac{4}{3}$.

The first several values of $g(p) \eqdef 
\frac{16 \Reg(\alpha, 2\alpha + 1) }
{(\ln(\disc_f/4))^2} 
$ are presented in the following
\begin{table}[h!]
\centering
\begin{tabular}{|c|c|}
\hline
$p$ & $g(p)$ \\
\hline
2 & 2.3005757200277737 \\
3 & 0.948774985707828 \\
5 & 1.7162543373046009 \\
7 & 0.824986119068219 \\
11 & 1.5902803808431747 \\
13 & 1.573545207927126 \\
17 & 1.5508005143276846 \\
19 & 1.5425857167054255 \\
23 & 1.529835319790773 \\
29 & 1.5163082966042292 \\
31 & 1.5127547443392193 \\
37 & 1.5039633069049658 \\
41 & 1.4992465878896728 \\
43 & 1.4971456295557026 \\
47 & 1.4933611793019335 \\
53 & 1.4885186045113814 \\
59 & 1.484436992267399 \\
61 & 1.483211643289589 \\
67 & 1.4798674282030506 \\
71 & 1.4778740553564413 \\
73 & 1.4769381932859227 \\
79 & 1.4743421857047407 \\
83 & 1.4727660220987417 \\
89 & 1.4705979224908206 \\
97 & 1.4680152465868979 \\
101 & 1.4668359850881518 \\
103 & 1.466271166102171 \\
107 & 1.4651872592939545 \\
109 & 1.4646667672702098 \\
113 & 1.4636655275160453 \\
127 & 1.4605230591156473 \\
\hline
\end{tabular}
\caption{Values of $p$ and $g(p)$}
\label{table:gp}
\end{table}
\newpage
which we computed using SageMath
\begin{lstlisting}[style=sagemath, caption=SageMath code]
for i, p in enumerate(Primes()):
    if i > 30:
        break
    R.<x> = PolynomialRing(QQ)
    P = x^3 + p^2 * x^2 - 1
    K.<a> = QQ.extension(P)
    rg = K.regulator()
    print(p, float(16 * rg * ln( p^6 - 27/4 )^(-2)))
\end{lstlisting}
\end{Rem}

\begin{Rem}
We just proved that the regulator is approximately 
\[
\ln(p) ( 3 \ln(p) + 2 \ln 2 )
,\] 
which is close to $3 \ln^2 p$ we wanted from the beginning.
\end{Rem}

\section{Centralizers}
\begin{Prop}
The centralizer of $\tilde A$ in $\SL$
is generated by $A, p A+\Id$.
$$Z_{\SL}(\tilde A) = \{ A^a (p A + \Id)^b \mid a, b \in \mathbb{Z} \} .$$ 
\end{Prop}
\begin{proof}
Since $\tilde A$ is regular, its centralizer in $ \Mat $ is $\C \left< \Id, A, A^2 \right> $.
Now, 
\[
\C \left< \Id, A, A^2 \right> \cap \SL 
\subset \mathbb{Z} 
\left< \Id, A, A^2 \right> = \mathbb{Z}[A]
.\] 
Indeed,
\begin{equation}
\Id = 
\begin{pmatrix}
1 & 0 & 0 \\
0 & 1 & 0 \\
0 & 0 & 1
\end{pmatrix}, 
A = 
\begin{pmatrix}
0 & 0 & 1 \\
1 & 0 & 0 \\
0 & 1 & -p^2
\end{pmatrix}, 
A^2 = 
\begin{pmatrix}
0 & 1 & -p^2 \\
0 & 0 & 1 \\
1 & -p^2 & p^4
\end{pmatrix}, 
\end{equation}
Considering the first matrix column we see that if a complex combination has integer coefficients, it is in fact integer combination.

Moreover, 
the centralizer of $\tilde A$ is a group, therefore it lies inside the multiplicative group of $\mathbb{Z}[A]$
\[
Z_{\SL}(\tilde A)  \subset 
\C \left< \Id, A, A^2 \right> \cap \SL 
\subset \mathbb{Z}[A]^*
.\] 
There is an isomorphism of $\mathbb{Z}$-algebras $\mathbb{Z}[A] \simeq \mathbb{Z}[x]/(f(x)) = \mathbb{Z}[\alpha]$.
Applying Proposition \ref{prop_multiplicative_group_structure} end the proof
\[
Z_{\SL}(\tilde A) 
\subset \mathbb{Z}[A]^* =\{ \pm A^a (p A + \Id)^b \mid a, b \in \mathbb{Z} \}.\] 
\end{proof}

We are to study the centralizer of $\tilde A$ in $\SLp$.
\[
Z_{\SLp}(\tilde A) \subset 
Z_{\SL}(\tilde A) \cong \mathbb{Z}^2
.\] 

\begin{Prop}
	The index of the one centralizer inside the other \emph{does not depend on $p$} and equals $3$.
\end{Prop}
\begin{proof}
As $A$ is equivalent to the cyclic permutation matrix $\pmod p$ it follows $A^3 \in \SLp$.

Thus the smaller centralizer is generated $A^3, \tilde A$
\[
Z_{\SLp}(\tilde A)
= \{ A^{3a} (p A + \Id)^b \mid a, b \in \mathbb{Z} \}
,\] 
and it's clear that the index is $3$.
\end{proof}




\bibliography{mybib}
\bibliographystyle{plain}

\end{document}
