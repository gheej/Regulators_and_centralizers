\documentclass[a4paper]{article}

\usepackage[utf8]{inputenc}
\usepackage[T1]{fontenc}
\usepackage{textcomp}
\usepackage[english]{babel}
\usepackage{amsmath, amssymb, amsthm, mathtools}
\usepackage{hyperref} % clickable links in the document
\usepackage{url} % clickable links in the document
\usepackage{enumitem} % (a), (b), etc labels

\usepackage{xcolor} % support of \red text
\newcommand{\red}{\textcolor{red}} % support of \red text

\newtheorem{Thm}{Theorem}[section]
\newtheorem{Lem}[Thm]{Lemma}
\newtheorem{Def}[Thm]{Definition}
\newtheorem{Cor}[Thm]{Corollary}
\newtheorem{Prop}[Thm]{Proposition}
\newtheorem{Rem}[Thm]{Remark}
\newtheorem{Ex}[Thm]{Example}
\newtheorem{Cla}[Thm]{Claim}
\newtheorem{Exer}[Thm]{Exercise}
\newtheorem{Prob}[Thm]{Problem}

\newcommand{\embeds}{\hookrightarrow}
\newcommand{\projects}{\twoheadrightarrow}
\newcommand{\eqdef}{\stackrel{\mathrm{def}}{=}}

\DeclareMathOperator{\assign}{\coloneqq}        % Assingment operator
\newcommand{\R}{\mathbb{R}}        % Real
\newcommand{\Q}{\mathbb{Q}}        % rationals
\newcommand{\C}{\mathbb{C}}        % Complex
\renewcommand{\P}{\mathbb{P}}        % Projective
\newcommand{\A}{\mathbb{A}}        % Affine
\renewcommand{\O}{\mathcal{O}}        % Structure sheaf
\newcommand{\Id}{\mathbf{Id}}        % Identity
\newcommand{\SL}{\mathbf{SL}_3(\mathbb{Z})}        % SL_3(Z)
\newcommand{\GL}{\mathbf{GL}_3} % GL_3
\newcommand{\PGL}{\mathbf{GL}_3} % PGL_3
\newcommand{\Mat}{\mathbf{Mat}_3(\mathbb{C})}        % Mat_3( C )
\newcommand{\MatZ}{\mathbf{Mat}_3(\mathbb{Z})}        % Mat_3( ZZ )
\newcommand{\SLp}{\Gamma_p}        % a congruence subgroup
\newcommand{\phip}{\phi_p}        % the golden ratio mod(p)
\newcommand{\hence}{\Rightarrow}        % hence
\newcommand{\minpoly}{\mathrm{minpoly}}        % minimal polynomial

\DeclareMathOperator{\Spec}{Spec}        % Spec
\DeclareMathOperator{\Proj}{Proj}        % Proj
\DeclareMathOperator{\Reg}{Reg}        % Regulator
\DeclareMathOperator{\disc}{disc}        % discriminant


\begin{document}
\title{A sample computation}	
\author{Altan Erdnigor}
\maketitle

\tableofcontents

\section{Notation}
\begin{itemize}
\item $p$ a prime number.
\item $\SL$ the special linear group over $\mathbb{Z}$.
\item $\SLp$ the $p$th congruence subgroup of $\SL$.
\item $Z_G(x)$ the centralizer of $x \in G$.
\item 
\begin{equation}
\label{matrix_small}
A = 
\begin{pmatrix}
0 & 0 & 1 \\
1 & 0 & p+2 \\
0 & 1 & 2p
\end{pmatrix}
\in \SL
\end{equation}
\item 
\begin{equation}
\label{matrix_big}
\tilde A = \Id + p A =
\begin{pmatrix}
1 & 0 & p \\
p & 1 & 2 p + p^2 \\
0 & p & 1 + 2 p^2
\end{pmatrix}
\in \SLp
\end{equation}

\item If $C$ is a matrix, $\chi_C(\lambda) \assign \det(\lambda \Id - C)$ is the characteristics polynomial.
\item $ f(t) \assign \chi_A(t) = t^3 - 2p t^2 - (p + 2) t - 1$.

\end{itemize}

\section{Intro}
Recall the Lucas primes (see \url{https://en.wikipedia.org/wiki/Lucas_number#Lucas_primes}, \url{https://t5k.org/top20/page.php?id=48})
\begin{equation}
	\label{the_primes_sequence}
2, 3, 7, 11, 29, 47, 199, 521, 2207, 3571, 9349, 3010349, 54018521, 370248451, 6643838879, \ldots
\end{equation}
In this notes we establish the following results:
\begin{enumerate}
\item
The regularor of $\tilde A$
grows as $\approx \ln^2(p)$. 

That is, $\Reg ( \Q(\alpha)/\Q ) \approx \ln^2(p) $ for $\alpha$ a root of $\chi_{\tilde A}$.

\item
The index of the centralizers
$ [Z_{\SL}(\tilde A) : Z_{\SLp}(\tilde A)] $
grows as $O(\ln p)$ for $p$ a Lucas prime.
\end{enumerate}
This might be interesting.
\section{Regulator}
We refer to Keith Conrad's write-up on Dirichlet's unit theorem and regulators \cite{conraddirichlet} for the definitions.
The current proof mimics the proof of Theorem 5.12 of Conrad.

Notice that 
\begin{multline}
	\label{computation_char_poly_of_a_tilde}
\chi_{\tilde A}(\lambda) 
= \det(\lambda \Id - \tilde A) \\
= \det(\lambda \Id - (\Id + p A))
= \det((\lambda - 1) \Id - p A)\\
= p^3 \det(\frac{\lambda - 1}p \Id - A)
= p^3 \chi_A(\frac{\lambda - 1}p)
.\end{multline}
Hence adding the root of $\chi_{A}$ or $\chi_{\tilde A}$ result in the same field; therefore we reduce to showing that 
$\Reg ( \Q(\alpha)/\Q ) \approx \ln^2(p) $ for $\alpha$ a root of $\chi_{A}(t) = f(t) = t^3 - (2p t^2 + (p + 2) t + 1) $.

\begin{Lem}
$\Q(\alpha)/\Q$ is totally real of degree $3$ for primes $p \ne 2$.
\end{Lem}
\begin{proof}
$f(t)$ is irreducible over $\Q$; indeed, by the rational roots theorem it's sufficient to check $\pm 1$:
$$f(1) = - 3 p - 2, f(-1) = - p .$$

A simple computation shows that the discriminant of $f(t)$ equals 
\[
	\disc_f(p) = 4 p^4 - 12 p^3 + 4 p^2 - 24 p + 5
.\]
If $p > 4$ the discriminant  $\disc_f(p) > 0$ is positive.
Therefore the cubic extension is totally real.
\end{proof}

\begin{Prop}
	\label{prop_multiplicative_group_structure}
$\mathbb{Z}[\alpha]^{*} = \{ \pm \alpha^a (2\alpha + 1)^b \mid a, b \in \mathbb{Z} \} $.
\end{Prop}
Note that $\alpha, 2\alpha + 1$ are not necessarily fundamental units in $\Q(\alpha)/\Q$ as we don't claim that the ring of integers of $\Q(\alpha) / \Q$ coincides with $\mathbb{Z}[\alpha]$.
\begin{proof}
Note that $f(\alpha) = 0$ implies 
\begin{gather}
\alpha (\alpha^2 - 2p \alpha - (p + 2)) = 1
\\
\label{equation_units}
(1 + 2 \alpha) (1 + p \alpha) = \alpha^3
\end{gather}
It shows that $\alpha, 1 + 2 \alpha$ are indeed units.

Let $\alpha_1 > \alpha_2 > \alpha_3 \in \R$ be the three different roots of $f$.
We shall compute them approximately.
\begin{gather*}
\alpha_1 = 2 p + \frac{1}{2} + O\left(\frac{1}{p}\right), \\
\alpha_2 = - \frac{1}{ p } + O\left(\frac{1}{p^4}\right), \\
\alpha_3 = - \frac{1}{2} + O\left(\frac{1}{p}\right)
.\end{gather*}
% from sympy import *
% [ins] In [2]: x = symbols('x')
% [ins] In [3]: p = 10000
% [ins] In [12]: f = x**3 - 2 * p * x**2 - (p + 2) * x - 1
% [ins] In [13]: f = Poly(f)
% [ins] In [16]: nroots(f)
% Out[16]: [-0.499987498124648, -0.000100000000000100, 20000.5000874981]
\begin{Rem}




A computation shows that for $p = 10000$
we have
 \[
\alpha_1 = 20000.5000874981 ,
\alpha_2 = -0.000100000000000100 ,
\alpha_3 = -0.499987498124648 
.\] 
It is not important that $p$ is not a prime in this case as the estimate works for any sufficiently large $p$.
\end{Rem}

By the definition of the regulator we have
\begin{multline}
\Reg(\alpha, 2\alpha + 1) = 
\begin{vmatrix}
\ln |\alpha_1| 		& \ln |\alpha_2| \\
\ln |2 \alpha_1 + 1| 	& \ln |2 \alpha_2 + 1|
\end{vmatrix} 
\\
\approx
\begin{vmatrix}
\ln |2p + \frac{1}{2}| 	& \ln |\frac{-1}{p}| \\
\ln |4 p + 2| 	& \ln |\frac{-2}{p} + 1|
\end{vmatrix}
=
\ln(2p + \frac{1}{2})
(\ln(p - 2) - \ln(p))
+
\ln(4p + 2) \ln(p)
\\
=
\ln(2 p + \frac{1}{2}) \ln(p - 2) - \ln(2 p + \frac{1}{2}) \ln(p) + \ln(4p + 2) \ln(p)
.\end{multline}
Therefore $ \Reg(\alpha, 2\alpha + 1) > 0 $ for all prime $p$.

Hence $\alpha, 2\alpha + 1$ are independent units. 

It is left to prove that they are fundamental units in $\mathbb{Z}[\alpha]$.
By Corollary 5.9 from Conrad it is sufficient to check
\[
\frac{16 \Reg(\alpha, 2\alpha + 1) }
{(\ln(\disc_f/4))^2} 
< 2
.\] 
Substituting, we obtain
\[
\frac{16 \Reg(\alpha, 2\alpha + 1) }
{(\ln(\disc_f/4))^2} 
\approx
\frac{16 (
\ln(2 p + \frac{1}{2}) \ln(p - 2) - \ln(2 p + \frac{1}{2}) \ln(p) + \ln(4p + 2) \ln(p)
) }
{(\ln(
(p^4 + 2 p^3 - 5 p^2 - 6 p - 23)
/4))^2} 
.\] 
Asymptotically, the latter equals
\[
\stackrel{p \to \infty}{\longrightarrow}
\frac{16 
\ln(p)^2
}
{(\ln(
p^4 ))^2} = 1
.\] 
Therefore it is $< 2$ for big enough $p$, QED.
\end{proof}
\begin{Rem}
One can do the estimate more carefully, but for now postpone it. 
\end{Rem}
\begin{Rem}
We just proved that the regulator is approximately 
\[
\ln(2 p + \frac{1}{2}) \ln(p - 2) - \ln(2 p + \frac{1}{2}) \ln(p) + \ln(4p + 2) \ln(p)
,\] 
which is close to $\ln^2 p$ we wanted from the beginning.
\end{Rem}


\section{Centralizers}
\begin{Prop}
	\label{proposition_centralizer_in_sl}
The centralizer of $\tilde A$ in $\SL$
is generated by $A, A+\Id$.
$$Z_{\SL}(\tilde A) = \{ \pm A^a (2 A + \Id)^b \mid a, b \in \mathbb{Z} \} .$$ 
\end{Prop}
\begin{proof}
Since $\tilde A$ is regular, its centralizer in $ \Mat $ is $\C \left< \Id, A, A^2 \right> $.
Now, 
\[
\C \left< \Id, A, A^2 \right> \cap \SL 
\subset \mathbb{Z} 
\left< \Id, A, A^2 \right>
.\] 
Indeed,
\begin{equation}
\Id = 
\begin{pmatrix}
1 & 0 & 0 \\
0 & 1 & 0 \\
0 & 0 & 1
\end{pmatrix}, 
A = 
\begin{pmatrix}
0 & 0 & 1 \\
1 & 0 & p+2 \\
0 & 1 & 2p
\end{pmatrix}, 
A^2 = 
\begin{pmatrix}
0 & 1 & 2p \\
0 & p+1 & 2p^2+4p+1 \\
1 & p & 4p^2+p+2
\end{pmatrix}, 
\end{equation}
Considering the first matrix column we see that if a complex combination has integer coefficients, it is in fact integer combination.

Moreover, 
the centralizer of $\tilde A$ is a group, therefore it lies inside the multiplicative group of $\mathbb{Z}[A]$
\[
Z_{\SL}(\tilde A)  \subset 
\C \left< \Id, A, A^2 \right> \cap \SL 
\subset \mathbb{Z}[A]^*
.\] 
There is an isomorphism of $\mathbb{Z}$-algebras $\mathbb{Z}[A] \simeq \mathbb{Z}[x]/(f(x)) = \mathbb{Z}[\alpha]$.
Applying Proposition \ref{prop_multiplicative_group_structure} end the proof
\[
Z_{\SL}(\tilde A) 
\subset \mathbb{Z}[A]^* =\{ \pm A^a (2 A + \Id)^b \mid a, b \in \mathbb{Z} \}.\] 
\end{proof}

We are to study the centralizer of $\tilde A$ in $\SLp$.
\[
Z_{\SLp}(\tilde A) \subset 
Z_{\SL}(\tilde A) \cong \mathbb{Z}^2
.\] 
In general this subgroup can be difficult to describe; 
that leads to cosidering \emph{Lucas primes}.

\subsection{Lucas primes give small degree $\log(p)$}

\begin{Prop}
Let $p$ be a Lucas prime.
Let $k$ be the integer part of $\log_{\phi} p$ where $\phi$ is the golden ratio.

The centralizer $Z_{\SLp}(\tilde A)$ contains $\tilde A$ and $A^{4k}$.
\end{Prop}
\begin{proof}
The only thing to prove is that 
$A^{4k}  \in \SLp$.

It suffices to show that the eigenvalues of $A \pmod p$ are $4k$-th roots of unity.
Computing
\[
\chi_{A}(t) = f(t) \equiv t^3 - 2 t - 1 = (t + 1) (t^2 - t - 1) \pmod p
,\] 
shows that it is left to work with the golden ratio in $\mathbb F_p$ which we denote by $\phip$.
That is, $\phip \in \mathbb F_{p^2}$ satisfies $\phip^2 - \phip - 1 = 0$.


By the definition of a Lucas number we have 
\begin{equation}
	\label{eq_lucas_prime_definition}
p = \phi^k + (-\phi)^{-k} 
\end{equation}
where $\phi$ is a root of $x^2 - x - 1$.

The RHS of \eqref{eq_lucas_prime_definition} being invariant under the change $\phi \to (-\phi)^{-1}$ manifests it as a symmetric polynomial in the roots of $x^2 - x - 1$, thus having a presentation
\[
\phi^k + (-\phi)^{-k} = P( \phi, (-\phi)^{-1})
,\] 
where $P$ is a \emph{universal} polynomial.
This observation justifies that the Equation \eqref{eq_lucas_prime_definition} can be taken modulo $p$ to have the form
\[
0 = \phip^k + (-\phip)^{-k} 
,\] 
which implies 
\[
1 = \phip^{4k} 
.\] 
\end{proof}

\begin{Thm}
	The index of the centralizers is bounded by $4 \log_{\phi} p$
\[
	[Z_{\SL}(\tilde A) : Z_{\SLp}(\tilde A)] \le 4 \log_{\phi} p
.\] 
In particular, it grows as $O(\ln p)$.
\end{Thm}
\begin{proof}
Identify $Z_{\SL}(\tilde A) \cong \mathbb{Z}/2\mathbb{Z} \oplus \mathbb{Z}^2$ as in Proposition \ref{proposition_centralizer_in_sl}.

Observe that by \eqref{equation_units} we have
\[
1 + p \alpha = \alpha^3 (2 \alpha + 1)^{-1} 
.\] 

By the previous Proposition 
$Z_{\SLp}(\tilde A)$ contains $\binom{3}{-1}, \binom{4k}{0}$;
Clearly, the index
\[
\mathbb{Z} \left< \binom{3}{-1}, \binom{4k}{0}\right> \subset \mathbb{Z}^2
,\] 
equals $4k \approx 4 \log_{\phi} p$ and the index 
$ [Z_{\SLp}(\tilde A) : Z_{\SL}(\tilde A)]$
has to divide it.
\end{proof}

\subsection{Bounds on the degree $log(p) \prec \deg \prec p$}
Motivated by geometric considerations, we call the degree $\deg$ the index $[Z_{\SLp}(\tilde A) : Z_{\SL}(\tilde A)]$.

\begin{Prop}
We have $O(log(p)) \le \deg \le O(p)$.
\end{Prop}

\begin{proof}
The proof of the previous Theorem shows that $Z_{\SL}(\tilde A)$ is generated by  $\alpha^3(2 \alpha + 1)^{-1}, \alpha^{\deg}$.

So, the goal is to show that the multiplicative order of $\phi_p$ is bounded by $\log(p)$ from below and by $p$ from above.

First let us show $\log(p) < \deg$.
Indeed, lifting $\phi_p$ to $\phi$ we notice that 
\[
\phi_p^k = 1 \hence (- \phi_p^{-1} )^{k} = (-1)^k \hence \phi^k + (- \phi^{-1} )^{k} - (1 + (-1)^k) \in p \mathbb{Z}
.\] 
Thus the degree $\deg > \log(p) - \epsilon$ for $p$ big enough and some small $\epsilon$.

Finally, we shall show $\deg \le O(p)$.
What we show in reality is that if $\sqrt 5 \in \mathbb F_p$, the degree $\deg$ divides $p-1$, $\deg | p-1$ and if $\sqrt 5 \notin \mathbb F_p$, $\deg | 2 p + 2$.

Indeed, the first part is clear by Fermat's little theorem.
To show the second, let's decompose 
\[
x^2 - x - 1 = (x - \alpha_1)(x - \alpha_2)
,\] 
where $\alpha_1, \alpha_2 \in \mathbb F_{p^2}$.
The Frobenius automorphism $\mathrm{Frob}: t \mapsto t^p$ acts on the roots $\alpha_1, \alpha_2$ by permuting them; thus we have $\alpha_1^p = \alpha_2 = -1/\alpha_1$ implying $\alpha_1^{2p+2} = (-1)^2 = 1$.
\end{proof}



\section{Enter family}
Let $f_{a,b}(t) = t^3 - a p t^2 - (b p + a) t - b$.
Here $a, b \in \mathbb{Z} $.
Note that $f_{2,1}(t) = f(t)$ is the characteristic polynomial of $A$.

We can now define
\begin{equation}
\label{matrix_small_family}
A_{a, b} = 
\begin{pmatrix}
0 & 0 & b \\
1 & 0 & bp+a \\
0 & 1 & ap
\end{pmatrix}
\in \MatZ,
\end{equation}
\begin{equation}
\label{matrix_big_family}
\tilde A_{a, b} = \Id + p A_{a, b} =
\begin{pmatrix}
1 & 0 & b p \\
p & 1 & b p^2 + a p \\
0 & p & 1 + a p^2
\end{pmatrix}
\in \SLp.
\end{equation}
Here $\tilde A_{a, b}$ is indeed invertible thanks to 
\begin{multline}
	\label{computation_char_poly_of_a_tilde_family}
	\det({{\tilde{A}}_{a,b}})
= p^3 \det \left(\frac{1}p \Id + A_{a,b}\right) 
= - p^3 \det \left(\frac{-1}p \Id - A_{a,b}\right) \\
= - p^3 f_{a,b}\left(\frac{-1}p\right)
= 1 + a p^2 - (b p + a) p^2 + b p^3
= 1
.\end{multline}
Actually, a generic element in $\SLp$ is conjugate to $\tilde A_{a, b}$.
\begin{Prop}
An element $\tilde B \in \SLp$ is either
\begin{enumerate}
	\item $\Id$;
	\item conjugate to 
		\[
		\begin{pmatrix}
		1 & p & 0 \\
		0 & 1 & 0 \\
		0 & 0 & 1
		\end{pmatrix}
		\] 
		over $\mathbb{Q}$;
	\item conjugate to $\tilde A_{a, b}$ over $\mathbb{Q}$ for some $a, b \in \mathbb{Z}$.
\end{enumerate}
% That is, for any $B \neq \Id \in \SLp$ we have
% \[
% B = C \tilde A_{a, b} C^{-1} 
% ,\] for some $a, b \in \mathbb{Z}, C \in \GL( \mathbb{Q} )$.
% Moreover, $a, b$ are uniquely defined and $C$ is unique as an element of $\PGL( \mathbb{Q} )$.
\end{Prop}
The proof goes by applying the Frobenius normal form over $ \mathbb{Q}$.
\begin{proof}
Define $B = \frac{1}{p}(\tilde B - \Id) \in \MatZ$.

There are three cases:
the minimal polynomial $\minpoly_B$ of $B$ is of degree $1, 2$ or $3$.
\begin{enumerate}
	\item $\deg \minpoly_B = 1$.  Hence $\tilde B$ is scalar, hence $\Id$;
	\item $\deg \minpoly_B = 2$.  Hence $ \frac{\chi_B}{\minpoly_B}$ is linear, thus dividing $\minpoly_B$; we obtain $\chi_B$ decomposes into a product of linear multiples over $ \mathbb{Z}$
	$$
	\chi_{B} (\lambda) = (\lambda - a)^2 (\lambda - b), a, b \in \mathbb{Z}.
	$$
	The condition $ - p^3 \chi_{B}( \frac{-1}{p}) = \chi_{\tilde B}(0) = 1$ now reads
	\[
	(1 + a p)^2 (1 + b p ) = 1
	,\] 
	thus $a, b = 0$ if $p > 2$ and $B$ is nilpotent.
	Therefore $\tilde B$ is conjugate to 
	\[
	\begin{pmatrix}
	1 & p & 0 \\
	0 & 1 & 0 \\
	0 & 0 & 1
	\end{pmatrix}
	.\] 
	\item
	The Frobenius normal form of 
	$B$ 
	% $\frac{1}{p}(B - \Id)$ 
	consists of the unique $3 \times 3$ block.  

	Pick $C \in \GL( \mathbb{Q} ) $ such that 
	$$
	C^{-1} B C =
	\begin{pmatrix}
	0 & 0 & r_0 \\
	1 & 0 & r_1 \\
	0 & 1 & r_2
	\end{pmatrix} \in 
	\mathbf{SL}_3(\mathbb{Q})
	.$$ 
	This matrix is actually integer $ C^{-1} B C \in \MatZ$.
	Indeed, the coefficients in the last column are the coefficients of the characteristic polynomial. 
	Therefore, $ C^{-1} \tilde B C \in \SLp $.

	We now show that 
	$ C^{-1} \tilde B C = \tilde A_{a, b}$.
	This follows from the equation $\det (C^{-1} \tilde B C) = 1$ which reads
	\[
	1 + r_2 p - r_1 p^2 + r_0 p^3 = 1
	,\] 
	implying
	 \[
	r_0 = b, r_1 = bp + a, r_2 = ap,
	\] 
	for some $a, b \in \mathbb{Z}$.
\end{enumerate}


\end{proof}

\section{The Unknown}
\begin{Prob}
What is the behavior of $\Reg, \deg $ for an arbitrary member of the family $\tilde A_{a, b}$?
\end{Prob}

TODO:
* Study the problem for fixed p and changing $a, b$;
here $a, b$ are such that $\Reg f_{a,b} < p/5$?

\[
\disc f_{a,b} 
= a^2*p^4*b^2 - 2*a^3*p^3*b + a^4*p^2 + 4*p^3*b^3 - 6*a*p^2*b^2 - 6*a^2*p*b + 4*a^3 - 27*b^2
.\] 

















\bibliography{mybib}
\bibliographystyle{plain}

\end{document}
