\documentclass[a4paper]{article}

\usepackage[utf8]{inputenc}
\usepackage[T1]{fontenc}
\usepackage{textcomp}
\usepackage[english]{babel}
\usepackage{amsmath, amssymb, amsthm, mathtools}
\usepackage{hyperref} % clickable links in the document
\usepackage{enumitem} % (a), (b), etc labels

\usepackage{xcolor} % support of \red text
\newcommand{\red}{\textcolor{red}} % support of \red text

\newtheorem{Thm}{Theorem}[section]
\newtheorem{Lem}[Thm]{Lemma}
\newtheorem{Def}[Thm]{Definition}
\newtheorem{Cor}[Thm]{Corollary}
\newtheorem{Prop}[Thm]{Proposition}
\newtheorem{Rem}[Thm]{Remark}
\newtheorem{Ex}[Thm]{Example}
\newtheorem{Cla}[Thm]{Claim}
\newtheorem{Exer}[Thm]{Exercise}
\newtheorem{Prob}[Thm]{Problem}

\newcommand{\embeds}{\hookrightarrow}
\newcommand{\projects}{\twoheadrightarrow}
\newcommand{\eqdef}{\stackrel{\mathrm{def}}{=}}

\DeclareMathOperator{\assign}{\coloneqq}        % Assingment operator
\newcommand{\R}{\mathbb{R}}        % Real
\newcommand{\Q}{\mathbb{Q}}        % rationals
\newcommand{\C}{\mathbb{C}}        % Complex
\renewcommand{\P}{\mathbb{P}}        % Projective
\newcommand{\A}{\mathbb{A}}        % Affine
\renewcommand{\O}{\mathcal{O}}        % Structure sheaf
\newcommand{\Id}{\mathbf{Id}}        % Identity
\newcommand{\SL}{\mathbf{SL}_3(\mathbb{Z})}        % SL_3(Z)
\newcommand{\Mat}{\mathbf{Mat}_3(\mathbb{C})}        % Mat_3( C )
\newcommand{\SLp}{\mathbf{SL}_3(p)}        % SL_3(p)
% \newcommand{\SLp}{\Gamma_p}        % a congruence subgroup

\DeclareMathOperator{\Spec}{Spec}        % Spec
\DeclareMathOperator{\Proj}{Proj}        % Proj
\DeclareMathOperator{\Reg}{Reg}        % Regulator
\DeclareMathOperator{\disc}{disc}        % discriminant


\begin{document}
\title{A sample computation}	
\author{Altan Erdnigor}
\maketitle

\section{Notation}
\begin{itemize}
\item $p$ a prime number.
\item $\SL$ the special linear group over $\mathbb{Z}$.
\item $\SLp$ the $p$th congruence subgroup of $\SL$.
\item $Z_G(x)$ the centralizer of $x \in G$.
\item 
\begin{equation}
\label{matrix_small}
A = 
\begin{pmatrix}
0 & 0 & 1 \\
1 & 0 & p+1 \\
0 & 1 & p
\end{pmatrix}
\in \SL
\end{equation}
\item 
\begin{equation}
\label{matrix_big}
\tilde A = \Id + p A =
\begin{pmatrix}
1 & 0 & p \\
p & 1 & p + p^2 \\
0 & p & 1 + p^2
\end{pmatrix}
\in \SLp
\end{equation}

\item If $C$ is a matrix, $\chi_C(\lambda) \assign \det(\lambda \Id - C)$ is the characteristics polynomial.
\item $ f(t) \assign \chi_A(t) = t^3 - p t^2 - (p + 1) t - 1$.

\end{itemize}

\section{Intro}
In this note we find an infinite series of primes  
\begin{equation}
	\label{the_primes_sequence}
59, 101, 167, 173, 211, 223, 271, 307, 317, 347, 449, 463, 593, 599, 607, 691, 719, 809, 821, 829, 853, \ldots
\end{equation}
with the following properties:
\begin{enumerate}
\item
The regularor of $\tilde A$
grows as $\approx \ln^2(p)$. 

That is, $\Reg ( \Q(\alpha)/\Q ) \approx \ln^2(p) $ for $\alpha$ a root of $\chi_{\tilde A}$.

\item
The index of the centralizers
\[
	[Z_{\SLp}(\tilde A) : Z_{\SL}(\tilde A)] < 3p
\] 
is bounded by $3p$.
\end{enumerate}
Combining them both one gets something about systoles which is of interest of geometric group theory?
\section{Regulator}
We refer to Keith Conrad's write-up on Dirichlet's unit theorem and regulators \cite{conraddirichlet} for the definitions.
The current proof mimics the proof of Theorem 5.12 of Conrad.

Notice that 
\begin{multline}
\chi_{\tilde A}(\lambda) 
= \det(\lambda \Id - \tilde A) \\
= \det(\lambda \Id - (\Id + p A))
= \det((\lambda - 1) \Id - p A)\\
= p^3 \det(\frac{\lambda - 1}p \Id - A)
= p^3 \chi_A(\frac{\lambda - 1}p)
.\end{multline}
Hence adding the root of $\chi_{A}$ or $\chi_{\tilde A}$ result in the same field; therefore we reduce to showing that 
$\Reg ( \Q(\alpha)/\Q ) \approx \ln^2(p) $ for $\alpha$ a root of $\chi_{A}(t) = f(t) = t^3 - (p t^2 + (p + 1) t + 1) $.

\begin{Lem}
$\Q(\alpha)/\Q$ is totally real of degree $3$ for primes $p \ne 2$.
\end{Lem}
\begin{proof}
$f(t)$ is irreducible over $\Q$; indeed, by the real roots theorem it's sufficient to check $\pm 1$:
$$f(1) = - 2 p - 1, f(-1) = - 1 .$$

A simple computation shows that the discriminant of $f(t)$ equals 
\[
	\disc_f(p) = p^4 + 2 p^3 - 5 p^2 - 6 p - 23 
.\]
If $p \ge 3$ the discriminant  $\disc_f(p) > 0$ is positive.
Therefore the cubic extension is totally real.
\end{proof}

\begin{Prop}
	\label{prop_multiplicative_group_structure}
$\mathbb{Z}[\alpha]^{*} = \{ \pm \alpha^a (\alpha + 1)^b \mid a, b \in \mathbb{Z} \} $.
\end{Prop}
Note that $\alpha, \alpha + 1$ are not necessarily fundamental units in $\Q(\alpha)/\Q$ as we don't claim that the ring of integers of $\Q(\alpha) / \Q$ coincides with $\mathbb{Z}[\alpha]$.
\begin{proof}
Note that $f(\alpha) = 0$ implies 
\begin{equation}
\label{equation_unit}
\alpha (\alpha + 1) (\alpha - p) = 1
\end{equation}
It shows that $\alpha, \alpha + 1$ are indeed units.

Let $\alpha_1 > \alpha_2 > \alpha_3 \in \R$ be the three different roots of $f$.
We shall compute them approximately using \eqref{equation_unit}.
\begin{gather*}
\alpha_1 = p + 1 + O\left(\frac{1}{p^3}\right), \\
\alpha_2 = - \frac{1}{ p } + O\left(\frac{1}{p^4}\right), \\
\alpha_3 = - 1 + \frac{1}{ p } + O\left(\frac{1}{p^3}\right)
.\end{gather*}
\begin{Rem}
A computation shows that for $p = 10000$
we have
 \[
\alpha_1 = 10001.000000009999 ,
\alpha_2 = -0.00010000000000010001 ,
\alpha_3 = -0.9999000099970006 
.\] 
It is not important that $p$ is not a prime in this case as the estimate works for any sufficiently large $p$.
\end{Rem}

By the definition of the regulator we have
\begin{multline}
\Reg(\alpha, \alpha + 1) = 
\begin{vmatrix}
\ln |\alpha_1| 		& \ln |\alpha_2| \\
\ln |\alpha_1 + 1| 	& \ln |\alpha_2 + 1|
\end{vmatrix} 
\\
\approx
\begin{vmatrix}
\ln |p + 1| 	& \ln |\frac{-1}{p}| \\
\ln |p + 2| 	& \ln |\frac{-1}{p} + 1|
\end{vmatrix}
=
\ln(p + 1)
(\ln(p - 1) - \ln(p))
+
\ln(p + 2) \ln(p)
\\
=
\ln(p + 1) \ln(p - 1) - \ln(p + 1) \ln(p) + \ln(p + 2) \ln(p)
.\end{multline}
Therefore $ \Reg(\alpha, \alpha + 1) > 0 $ for all prime $p$.

Hence $\alpha, \alpha + 1$ are independent units. 

By Corollary 5.9 from Conrad it is sufficient to check
\[
\frac{16 \Reg(\alpha, \alpha + 1) }
{(\ln(\disc_f/4))^2} 
< 2
.\] 
Substituting, we obtain
\[
\frac{16 \Reg(\alpha, \alpha + 1) }
{(\ln(\disc_f/4))^2} 
\approx
\frac{16 
\left(\ln(p + 1) \ln(p - 1) - \ln(p + 1) \ln(p) + \ln(p + 2) \ln(p)\right)
}
{(\ln(
(p^4 + 2 p^3 - 5 p^2 - 6 p - 23)
/4))^2} 
.\] 
Asymptotically, the latter equals
\[
\stackrel{p \to \infty}{\longrightarrow}
\frac{16 
\ln(p)^2
}
{(\ln(
p^4 ))^2} = 1
.\] 
Therefore it is $< 2$ for big enough $p$, QED.
\end{proof}
\begin{Rem}
One can do the estimate more carefully, but for now postpone it. 
Note that the sequence \ref{the_primes_sequence} begins from $p = 59$, thus we can restrict onto $p \ge 59$.
\end{Rem}
\begin{Rem}
We just proved that the regulator is approximately 
\[
\ln(p + 1) \ln(p - 1) - \ln(p + 1) \ln(p) + \ln(p + 2) \ln(p)
,\] 
which is close to $\ln^2 p$ we wanted from the beginning.
\end{Rem}


\section{Centralizers}
\begin{Prop}
	\label{proposition_centralizer_in_sl}
The centralizer of $\tilde A$ in $\SL$
is generated by $A, A+\Id$.
$$Z_{\SL}(\tilde A) = \{ \pm A^a (A + \Id)^b \mid a, b \in \mathbb{Z} \} .$$ 
\end{Prop}
\begin{proof}
Since $\tilde A$ is regular, its centralizer in $ \Mat $ is $\C \left< \Id, A, A^2 \right> $.
Now, 
\[
\C \left< \Id, A, A^2 \right> \cap \SL 
\subset \mathbb{Z} 
\left< \Id, A, A^2 \right>
.\] 
Indeed,
\begin{equation}
\Id = 
\begin{pmatrix}
1 & 0 & 0 \\
0 & 1 & 0 \\
0 & 0 & 1
\end{pmatrix}, 
A = 
\begin{pmatrix}
0 & 0 & 1 \\
1 & 0 & p+1 \\
0 & 1 & p
\end{pmatrix}, 
A^2 = 
\begin{pmatrix}
0 & 1 & p \\
0 & p+1 & p^2+p+1 \\
1 & p & p^2+p+1
\end{pmatrix}, 
\end{equation}
Considering the first matrix column we see that if a complex combination has integer coefficients, it is in fact integer combination.

Moreover, 
the centralizer of $\tilde A$ is a group, therefore it lies inside the multiplicative group of $\mathbb{Z}[A]$
\[
Z_{\SL}(\tilde A)  \subset 
\C \left< \Id, A, A^2 \right> \cap \SL 
\subset \mathbb{Z}[A]^*
.\] 
There is an isomorphism of $\mathbb{Z}$-algebras $\mathbb{Z}[A] \simeq \mathbb{Z}[x]/(f(x)) = \mathbb{Z}[\alpha]$.
Applying Proposition \ref{prop_multiplicative_group_structure} end the proof
\[
Z_{\SL}(\tilde A) 
\subset \mathbb{Z}[A]^* =\{ \pm A^a (A + \Id)^b \mid a, b \in \mathbb{Z} \}.\] 
\end{proof}

We are to study the centralizer of $\tilde A$ in $\SLp$.
\[
Z_{\SLp}(\tilde A) \subset 
Z_{\SL}(\tilde A) \cong \mathbb{Z}^2
.\] 
In general this subgroup can be difficult to describe; 
that leads to the following
\begin{Def}
	A prime number $p$ has property $Ch$ if $t^3 - t - 1 \in \mathbb F_p[t]$ splits completely, that is  
\[
t^3 - t - 1 = (t - \beta_1)(t - \beta_2)(t - \beta_3)
,\] 
for some $\beta_1, \beta_2, \beta_3 \in \mathbb F_p$.
\end{Def}
The sequence \eqref{the_primes_sequence} is exactly the sequence of primes with the property $Ch$.
Why is it infinite?
\begin{Prop}
The density of primes with the property $Ch$ is $\frac{1}{6}$; in particular, there are infinite number of such primes.
\end{Prop}
\begin{proof}
Let $E/ \Q$ be the splitting field of $g(t) = t^3 - t - 1$.
The discriminant of $g$ equals $-23$, hence $E / \Q$ is of degree $6$.

Recall that a prime $p$ in $\Q$ is said to \emph{split completely} iff 
\[
p \mathbb{Z} = \gamma_1 \gamma_2 \gamma_3 \gamma_4 \gamma_5 \gamma_6
,\] 
for $\gamma_i$ primes in $\O_E$.
\red{$p$ splits completely iff it satisfies $Ch$}.
\red{feels close to Cox \cite{cox2022primes} Proposition 5.11}.

By Chebotarev's density theorem the density of primes splitting completely in $E$ is one over the degree of the extension.
\end{proof}

\begin{Prop}
Let prime $p$ satisfy $Ch$.

The centralizer $Z_{\SLp}(\tilde A)$ contains $\tilde A$ and $(A + \Id)^{p - 1}$.
\end{Prop}
\begin{proof}
The only thing to prove is that 
$(A + \Id)^{p - 1} \in \SLp$.
It suffices to show 
\[
	%(\alpha + 1)^{p-1} - 1 \in p \mathbb{Z}[\alpha] ~ ~  (or \in p \O_{\Q(\alpha)}  ? )
	(x + 1)^{p - 1} - 1 \in (p, x^3 - x - 1)
.\] 
The polynomial vanishes at all but one residues from $\mathbb F_p$, namely at $\mathbb F_p \setminus \{ -1 \} $.

By the $Ch$ property $x^3 - x - 1$ has three distinct roots in $\mathbb F_p$; clearly $-1$ is not one of them.

Therefore the polynomial $ x^3 - x - 1 $ divides $ (x + 1)^{p - 1} - 1 $ in $\mathbb F_p[x]$, and we are done.
\end{proof}

\begin{Thm}
	The index of the centralizers divides $3 (p - 1) $. 
\[
	[Z_{\SLp}(\tilde A) : Z_{\SL}(\tilde A)] | 3 (p-1)
.\] 
In particular, it is smaller than $3 p$.
\end{Thm}
\begin{proof}
Identify $Z_{\SL}(\tilde A) \cong \mathbb{Z}/2\mathbb{Z} \oplus \mathbb{Z}^2$ as in Proposition \ref{proposition_centralizer_in_sl}.

Observe that 
\[
1 + p \alpha = \alpha^3 (\alpha + 1)^{-1} 
.\] 
Indeed, multiplying both sides by $(\alpha + 1)$ we obtain
\[
1 + (p + 1) \alpha + \alpha^2 = \alpha^3 
,\] 
which holds true by the definition of $\alpha$.

By the previous Proposition 
$Z_{\SLp}(\tilde A)$ contains $\binom{3}{-1}, \binom{0}{p-1}$;
Clearly, the index
\[
\mathbb{Z} \left< \binom{3}{-1}, \binom{0}{p-1}\right> \subset \mathbb{Z}^2
,\] 
equals $3 (p-1)$ and the index 
$ [Z_{\SLp}(\tilde A) : Z_{\SL}(\tilde A)]$
has to divide it.
\end{proof}


\section{Fantasies}
\subsection{An exercise in generating functions}

Let 
$$
f(x) = x^3 - a x^2 + b x - c
.$$ 
be a qubic polynomial.
Let $\alpha_1, \alpha_2, \alpha_3$ to be its roots.

We have
$$
t^3 f(t^{-1}) = 1 - a t + b t^2 - c t^3 = (1 - t \alpha_1)(1 - t \alpha_2)(1 - t \alpha_3)
.$$ 

The goal is to prove
\begin{Prop}
\label{prop_generating_function_roots_powers}
Define a generating series
\[
F(t, q) \assign \sum_{i = 0}^\infty 
(1 - t \alpha_1^i)(1 - t \alpha_2^i)(1 - t \alpha_3^i) q^i
.\] 
Then we have 
\[
F(t, q) =
\frac{1}{1 - q}
- t \frac{ 3 - 2 a q + b q^2 }{1 - a q + b q^2 - c q^3}
+ t^2 \frac{ 3 - 2 b q + a c q^2 }{ 1 - b q + a c q^2 - c^2 q^3 }
- t^3 \frac{1}{ 1 - c q}
.\] 
\end{Prop}
\begin{proof}
A direct computation.
\begin{multline}
F(t, q) 
= \sum_{i = 0}^\infty 
(1 - t \alpha_1^i)(1 - t \alpha_2^i)(1 - t \alpha_3^i) q^i
\\ = 
\sum_{i = 0}^\infty  q^i
- t \sum_{i = 0}^\infty (\alpha_1^i + \alpha_2^i +  \alpha_3^i) q^i
+ t^2 \sum_{i = 0}^\infty (\alpha_2^i\alpha_3^i + \alpha_1^i\alpha_3^i + \alpha_1^i\alpha_2^i) q^i
- t^3 \sum_{i = 0}^\infty (\alpha_1^i \alpha_2^i \alpha_3^i) q^i
\\ = 
\frac{1}{1 - q}
- t \left( \frac{1}{1 - \alpha_1 q} + \frac{1}{1 - \alpha_2 q}  + \frac{1}{1 - \alpha_3 q}\right) 
\\ + t^2  \left( \frac{1}{1 - \alpha_2 \alpha_3 q} + \frac{1}{1 - \alpha_1 \alpha_3 q}  + \frac{1}{1 - \alpha_1 \alpha_2 q}\right) 
- t^3 \frac{1}{1 - \alpha_1 \alpha_2 \alpha_3 q}
\\ = 
\frac{1}{1 - q}
- t \frac{{(1 - \alpha_2 q)( 1 - \alpha_3 q) + (1 - \alpha_1 q)( 1 - \alpha_3 q) + (1 - \alpha_1 q)( 1 - \alpha_2 q) }
}{(1 - \alpha_1 q)( 1 - \alpha_2 q)(1 - \alpha_3 q)}
\\ + t^2 \frac{{(1 - \alpha_1\alpha_3 q)( 1 - \alpha_1\alpha_2 q) + (1 - \alpha_2\alpha_3 q)( 1 - \alpha_1\alpha_2 q) + (1 - \alpha_2\alpha_3 q)( 1 - \alpha_1\alpha_3 q) }
}{(1 - \alpha_2\alpha_3 q)( 1 - \alpha_1\alpha_3 q)(1 - \alpha_1\alpha_2 q)}
\\ - t^3 \frac{1}{1 - \alpha_1 \alpha_2 \alpha_3 q}
\\ = 
\frac{1}{1 - q}
- t 
\frac{
	3 - 2 (\alpha_1 + \alpha_2 + \alpha_3) q + (\alpha_2\alpha_3 + \alpha_1\alpha_3 + \alpha_1\alpha_2) q^2
}{q^3 f(q^{-1} )}
\\ + t^2 \frac{
	3 - 2 (\alpha_2 \alpha_3 + \alpha_1 \alpha_3 + \alpha_1 \alpha_2 ) q + (\alpha_1 + \alpha_2 + \alpha_3) \alpha_1 \alpha_2 \alpha_3  q^2
}{
	1 - (\alpha_2 \alpha_3 + \alpha_1 \alpha_3 + \alpha_1 \alpha_2 ) q + (\alpha_1 + \alpha_2 + \alpha_3) \alpha_1 \alpha_2 \alpha_3 q^2 - \alpha_1^2 \alpha_2^2 \alpha_3^2 q^3
}
\\ - t^3 \frac{1}{1 - \alpha_1 \alpha_2 \alpha_3 q}
\\ = 
\frac{1}{1 - q}
- t \frac{ 3 - 2 a q + b q^2 }{1 - a q + b q^2 - c q^3}
+ t^2 \frac{ 3 - 2 b q + a c q^2 }{ 1 - b q + a c q^2 - c^2 q^3 }
- t^3 \frac{1}{1 - c q}
\end{multline}
\end{proof}

Apply Proposition  
\ref{prop_generating_function_roots_powers}
to 
$$g(t) = t^3 - t - 1.$$
Therefore
\[
G(t, q) = 
\frac{1}{1 - q}
- t \frac{ 3 - q^2 }{1 - q^2 - q^3}
+ t^2 \frac{ 3 + 2 q }{ 1 + q - q^3 }
- t^3 \frac{1}{1 - q}
.\] 







\bibliography{mybib}
\bibliographystyle{plain}

\end{document}
